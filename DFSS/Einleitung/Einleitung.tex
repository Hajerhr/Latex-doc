\noindent Im Rahmen von Design For Six Sigma werden statistische Methoden eingesetzt. In der Optimize-Phase werden Methoden der Statistik verwendet, um Prozesssicherheiten zu bewerten und Streuungen von Bauelementen und Prozessen zu einer Gesamttoleranz zu \"{u}berlagern. Die statistischen Methoden zur Optimierung des Produktes in der Entwicklungsphase sind Statistische Simulation und Robust Design sowie statistische Versuchsplanung und statistische Tolerierung. In der Verify-Phase wird die prognostizierten Fertigbarkeit und Zuverl\"{a}ssigkeit best\"{a}tigt. Dazu ist neben einem durchdachten Erprobungsplan ein Nachweis der Messf\"{a}higkeit von Messeinrichtungen in Labor und Fertigung notwendig. Auf Basis geeigneter Messeinrichtungen erfolgt f\"{u}r die Fertigungsprozesse eine statistische Prozesskontrolle (SPC). Das erforderliche Grundwissen im Bereich der Wahrscheinlichkeitstheorie und Statistik wird in Dokument vermittelt.\newline

\noindent Zu Beginn werden die Grundlagen der Wahrscheinlichkeitstheorie behandelt, um einen ersten Einstieg in die Statistik zu erm\"{o}glichen. Hierbei werden Mengenoperationen zur Beschreibung statistische Begebenheiten eingef\"{u}hrt und der Begriff der Wahrscheinlichkeit nach Laplace und Kolmogoroff erl\"{a}utert.\newline

\noindent Nach den Grundlagen der Wahrscheinlichkeitstheorie werden dem Leser die Grundlagen vorgestellt, die zur Wahrscheinlichkeitsrechnung mit einer Variablen erforderlich sind. Diese Aufgaben werden als univariate Aufgaben bezeichnet. Ausgehend von der beschreibenden Statistik wird \"{u}ber die univariate Wahrscheinlichkeitstheorie gezeigt, wie univariate Stichproben beurteilt werden k\"{o}nnen. Der Leser wird dabei in die Lage versetzt, von einer vorliegenden Stichprobe auf die Grundgesamtheit zu schlie{\ss}en und mittels Konfidenzintervallen deren Genauigkeit statistisch zu beschreiben. Au{\ss}erdem werden die Prinzipien von Hypothesentests und deren Anwendung erl\"{a}utert.\newline

\noindent Im n\"{a}chsten Schritt wird das Wissen auf multivariate Statistik erweitert. Es wird erkl\"{a}rt, wie mehrdimensionale oder multivariate Stichproben sowohl grafisch als auch mit Kenngr\"{o}{\ss}en dargestellt werden k\"{o}nnen. Mit der Varianzanalyse wird das Streuverhalten eines Systems in Abh\"{a}ngigkeit von Merkmalen beschrieben. Ausgehend von der einfaktoriellen Varianzanalyse wird die mehrfaktorielle Varianzanalyse eingef\"{u}hrt und an Beispielen angewandt. Darauf aufbauend wird die Korrelationsanalyse eingef\"{u}hrt. Abschlie{\ss}end wird in dem Kapitel Regressionsanalyse gezeigt, wie multivariate Stichproben mathematisch durch Funktionen approximiert werden k\"{o}nnen.\newline

\noindent In die Darstellung sind viele Hinweise von Kollegen und Studierenden eingeflossen, f\"{u}r die ich mich an dieser Stelle herzlich bedanken m\"{o}chte. F\"{u}r weitere Hinweise bin ich jederzeit dankbar.\newline

\noindent Karlsruhe, 09.07.2020
